\documentclass[12pt]{article}

\usepackage{graphicx}
\usepackage[utf8]{inputenc} 
\usepackage[turkish]{babel}
\usepackage{amsmath}
\usepackage{hyperref}

\begin{document}

\title{Introduction to \LaTeX{}}
\author{Vefik Fırat Akman}

\maketitle

\begin{abstract}
Bİl101 HW09
\end{abstract}

\section{Reinforcement learning (pekiştirmeli öğrenme) nedir ve diğer makine
öğrenmesi yöntemlerindenfarkı nedir? Yarım sayfada kendi cümlelerinizle açıklayın.}
Reinforcement Learning, Machine Learning’in alt dalıdır. Bilgisayar doğru 
sonuç buldukçaödüllendilir. Yanlış durumlarda ise cezalandırılır.
Bu durum insanın öğrenme aşamasına benzetilir. Örnek olarak ise genellikle 
bir bebeğin hayatında ilk defasıcak bir çaya dokunması verilir. Bebek dokunmaması 
gereken sıcak çaya dokununcaeli yanacak yani cezalandırılmış olacaktır. 
Doğru şeylerde misal ilk adımlarınıattığında da ailesi tarafından sevgiyle
ödüllendirelecektir. Bu sistem de insanın bu doğası taklit edilmiştir.

\section{ Görüntü işleme, 2 boyutlu 3 boyutlu grafik tekniklerinin birbirinden
farkı nedir? 3 boyutlu grafik işlemenin 3 temel adımını açıklayınız. Yarım sayfada
kendi cümlelerinizle açıklayın.}
Görüntü işleme görüntü verilerinin içeriğini bilgisayar yardımıyla tespi etmeye
ya da amaca uygun olarak değiştirmeye yarayan sistemlerin genel adıdır.
2B Grafikte kullanıcı yaratılan objeyi (grafiği) tek açıdan görür. Yani objenin
de tek açıdan görüntüsü oluşturulur. 3D obje de kullanıcı istediği açıdan
objeye bakabilir. Bu yüzden objenin her açıdan görüntüsü oluşturulmalıdır.
1.Nokta tanımlaması
2.Doğru parçası tanımı
3.Yüzey tanımlanması
\end{document}